\documentclass[a4paper, 10pt]{exam}
\usepackage[margin=1in]{geometry}

\usepackage[utf8]{inputenc}
\usepackage{amssymb, amsmath, graphicx, multicol}
\usepackage{pgf,tikz, graphicx}
\usepackage{mdframed}
\usepackage{mathrsfs}

\everymath{\displaystyle}
\printanswers
\newmdtheoremenv{proof}{Proof}

\begin{document}
\fbox{\textbf{Math 140 - HW 1.1}} 
\textbf{REGALARIO, Jeremiah Daniel A. | 2022-20670}
\begin{questions}
    \question For a geometry model $G$, consider these three statements:
    \begin{enumerate}
      \item[(i)] $\mathscr{P} (G)$ is finite.
      \item[(ii)] $\mathscr{L} (G)$ is finite.
      \item[(iii)] $\mathscr{P}(G) \cup \mathscr{L} (G)$ is finite.
    \end{enumerate}
    
    \begin{enumerate}
      \item[(a)] Show that (i) and (iii) are equivalent. As part of your proof for (i) $\Rightarrow$ (iii), calculate the maximum possible cardinality of $\mathscr{P}(G) \cup \mathscr{L}(G)$ given that the cardinality of $\mathscr{P}(G)$ is finite, say $k \in \mathbb{N} \cup \{0\}$.
        \begin{proof}
            WTS: $(i) \iff (iii)$\\
           $ (\Rightarrow)$ Suppose $\mathscr{P} (G) $ is finite. Then, $|\mathscr{P}(G)| = k \in \mathbb{N}\cup \{0\}.$ 
           It follows that $\mathscr{P}(G) \cup \mathscr{L} (G) \subseteq \mathscr{P}(G) \cup 2^{\mathscr{P} (G)} \setminus \{\varnothing\}.$ Furthermore,
           \begin{align*}
               |\mathscr{P}(G) \cup \mathscr{L} (G)| &\leq |\mathscr{P}(G) \cup 2^{\mathscr{P} (G)} \setminus \{\varnothing\}| \\
               &= |2^{\mathscr{P} (G)} \setminus \{\varnothing\}| \\
               &= 2^{|\mathscr{P} (G)|} - 1  \\
               &= 2^k - 1
           \end{align*}
            Therefore, the maximum possible cardinality of $\mathscr{P}(G) \cup \mathscr{L} (G)$ is $\boxed{2^k - 1} \in \mathbb{N} \cup \{0\}. $\\
            Hence, $\mathscr{P}(G) \cup \mathscr{L} (G)$ is finite. \\
            
           $ (\Leftarrow)$ Suppose $\mathscr{P}(G) \cup \mathscr{L} (G)$ is finite. Suppose further, on the contrary, that $\mathscr{P} (G)$ is infinite. Then, $\mathscr{P}(G) \cup \mathscr{L} (G) \supseteq \mathscr{P}(G)$ is also infinite, a contradiction. Hence, $\mathscr{P} (G)$ is finite. \\

           Indeed, $\mathscr{P} (G)$ is finite is equivalent to $\mathscr{P}(G) \cup \mathscr{L} (G)$ is finite. $\blacksquare$
        \end{proof}
      \item[(b)] Demonstrate that (i) and (ii) are not equivalent, by giving an example of a geometry model $G$ with $\mathscr{L}(G) \neq \varnothing$ in which either (i) or (ii) holds but the other does not.
        \begin{proof}
            WTS: $(i) \not\equiv (ii)$ \\
            Consider the geometry model $G$ with $\mathscr{P} (G) = \mathbb{R}^2$ and $\mathscr{L} (G) =\{\{(x,y) \in \mathbb{R}^2 \mid x + y = 0 \}\}$. Then, $\mathscr{L} (G)$ is finite with $|\mathscr{L} (G)| = 1 \in \mathbb{N} \cup \{0\}$ but $\mathscr{P} (G)$ is infinite. \\
             
            Hence, $\mathscr{P} (G)$ is finite is not equivalent to $\mathscr{L} (G)$ is finite. $\blacksquare$
        \end{proof}

    \end{enumerate}
    
\end{questions}

\end{document}
